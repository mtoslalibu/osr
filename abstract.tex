Developers use logs to diagnose performance problems in distributed
  applications.  However, it is difficult to know a priori where logs
  are needed and what information in them is needed to help diagnose
  problems that may occur in the future.  We present the
  \underline{V}ariance-driven \underline{A}utomated
  \underline{I}nstrumentation \underline{F}ramework (\STAIF{}), which
  runs alongside distributed applications.  In response to
  newly-observed performance problems, \STAIF{} automatically searches
  the space of possible instrumentation choices to enable the logs
  needed to help diagnose them. 
  To work, \STAIF{} combines distributed tracing (an enhanced form of logging) with insights
  about how response-time variance can be decomposed on the
  critical-path portions of requests' traces.  We evaluate \STAIF{} by
  using it to localize performance problems in OpenStack.  We
  show that \STAIF{} can localize problems related to slow code paths,
  resource contention, and problematic third-party code while enabling
  only 3-34\% of the total tracing instrumentation.