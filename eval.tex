\mert{Removed all experiments. Only kept intro paragraph below and three case studies from openstack}

\staif{} serves as a fundamental step for developers to
diagnose performance problems, localizing them to a specific area of
the system.  We envision that developers will use \staif{} in two ways;
1) match anomalous traces (e.g., slow requests) to the hypothesis
forest to figure out the performance hypotheses, 2) query the
hypothesis forest by the request type to inquire about the sources of
performance problems.

\textcolor{red}{
This section presents three different case studies of how we used \STAIF{} to
identify various performance problems in OpenStack. Openstack is a widely-used distributed application for
managing clouds.  We use the OpenStack Stein release.  Our cluster
consists of 9 Compute and 1 Controller node.}

\noindent\textbf{Unpredictable performance of \textsc{VM Delete} requests.}
In OpenStack, instances can be deleted using the \textsc{VM Delete} command. 
We find that some \textsc{VM Delete} requests take 28 seconds, where the mean latency is around 20.
We query the hypothesis forest by the slowest request.
\staif{} finds the delete groups have high CV values (0.18) to begin with.
In exploration of variance, \staif{} enables three tracepoints and 
it localizes the high variance to $unplug\_os\_vif$ function. %with problematic group having CV of $0.12$.
This function constitutes 85\% of the variance and 42\% of response time in this request type.
Our analysis from there leads us to a bug report about the performance of this function (\cite{vifplug1,vifplug2}). %, which complains about this library and function.
For this problem, \staif{} helps diagnose a performance
bug due to a delay in a function. 


\noindent\textbf{Lack of instrumentation in the code. }
All instances on OpenStack can be listed using the command \textsc{VM List}.
Matching the slowest trace to the hypothesis forest shows that the request’s latency emanates from three edges. 
This trace's group shows high CV ($0.2$), and the enabled tracepoints constitute 63\% of all variance and 60\% of the latency.
We further examine the code corresponding to those three edges and find the following;
1) two edges ($keystone\_post \& get$) correspond to where identity service (keystone) is utilized for authentication token,
2) the third edge corresponds to a function ($get\_all$) that constitutes 2000 LoC and performs numerous DB lookups to get every instance, including deleted ones.
We corroborate these findings in the bug reports (\cite{listslow1,listslow2}),
 which state that {VM List} experiences latency variations due to  a) the token table getting large in identity service, and 
 b) the function not being able to scale well with the number of VMs and users.
In this case, \staif{} helps diagnose performance problems by isolating latency to (1) a specific service and operation and (2) an inefficient function. 
The latter case also provides an insight to developers as inefficient tracing (i.e., more tracepoints can be added to the 2000 LoC).

\noindent\textbf{Explaining the variance with key/value pairs.}  
% We re-inject the resource-contention problem described
% in Section ~\ref{sec:motivation:guiding_principles}.  
% We re-inject a problem to show how \staif{} can help debug a resource contation issues.
In OpenStack \textsc{VM\_CREATE} requests finish creating new OpenStack VMs very slowly. 
The root cause is that the
number of concurrent servers that can be created within OpenStack's Nova Service is limited to ten by default. 
Additional \textsc{VM\_CREATE} requests are forced to wait on a semaphore.

\staif{} finds that \textsc{VM Create} groups have high CV ($0.21$) initially.
\staif{} finds that the highest-variance edge constitutes
$96\%$ of the variance and 73\% of the response time.  The first
tracepoint of the highest variance edge ($\_build\_semaphore\_start$)
is where a nova-compute manager semaphore is acquired.
Further, \staif{} finds a high correlation between a key/value pair
exposed by the tracepoint (i.e., the number of simultaneous VM
creations) and request's latency (R=$0.85$ w/ p-val $10^{-5}$).